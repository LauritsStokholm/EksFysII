This paper is written as the last of four mandatory repports during the course
\emph{Experimental Physics II}. In the experiment we will be working with
spectroscopy. The spectra of different light bulbs (a diode based, a halogen
and an energy saving bulb) will be investigated and compared to the solar
spectrum. In addition to this, several of the distinct absorption lines (often
refered to as Fraunhofer lines), will be determined and the corresponding molecules
identified. Furthermore, the spectrum of a diode laser will be recorded above and below
threshold--operation as a function of intensity. Lastly three
cuvettes of three different molecules will be identified, by investigating
their absorbance.

\fxnote{THE RESULTS}
