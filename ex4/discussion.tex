\section{Discussion}

\subsection{Element lamps}

We have neglected the hyperfine structure as the database does not take it into acquaintance. The main reasons for errors in this experiment is our ability to determine the peaks in oceanwiev, the placement of our photocensor and other light sources. Ionization of the filament could also contribute to minor errors. 

\subsubsection{Xe}

Xe has Z=54 and its ground state configuration is $1s^2 2s^2 2p^6 3s^2 3p^6 3d^10 4s^2 4p^6 4d^10 5s^2 5p^6$. For simplicity we only noted the two last terms, as the are the electrons with lowest binding energy and are therefore the most likely to be excited. Furthermore, when a sufficient voltage is applied to the lamp, the gas becomes ionized and thus gets a free electron, that it did not have before due to it being a noble gas. Looking at our Xenon spectrum, we see that the most prominent lines with the highest relative intensity are 472 nm, 823 nm, 882 nm, 894 nm, 895 nm. Most of the transitions are from the $6s$ to the $6p$ shell, which is a shift to the subshell one level higher. The 472 nm transitions is however between to shells the $6s$ and the $7p$. It could also be noted that none of our jumps are from the ground state of Xe, but rather from an excited state caused by the voltage over the gas. Furthermore the peaks lies primarily within the 450-480 nm and 800-900 nm range which means that Xe radiates in the blue and infrared part of the spectrum. '

\subsubsection{Hg}

Hg has atomic number Z=80 and has ground state configuration $1s^2 2s^2 2p^6 3s^2 3p^6 3d^10 4s^2 4p^6 4d^10 4f^{14}5s^2 5p^6 5d^10 6s^2 {}^1S_0$,

where ${}^1S_{0}$ is an electronic configuration with total spin $S=0$ and total angular momentum $J=0$.\\

The prominent lines in the spectrum are 313 nm, 405 nm, 435 nm and 545 nm. All of these transitions are from the $6p$ subshell, where the 313 nm is a transition to the $6d$ shell, and the 405, 435 and 545 nm lines are to the $7s$ shell. Here the energy required to go from the p to the d subshell of Hg is greater than the one required to make a transisition to the $7s$ shell. Once again all of our transitions are from an already excited state due to the voltage over the gas. We can also note that the transition energy to the subshell $6d$ is actually greater than that to an entirely different shell $7s$. For alkali atoms this could be explained by the energies being inversly proportional to the effective primary quantum number, thus there will be smaller and spacing between the energies of the outer shells as the atom number increases. This spacing could be so small, that the fine structure eventually would lead to energy splittings larger than those of the actual primary quantum number. Hg is not an alkali metal, but by assuming that it is generally true that the outer shells are closer spaced energy-wise for larger atoms, this could explain our observations. Since we observe the most peaks within the 300-370 nm and the 400-600 nm range, we can conclude that Hg radiates within the ultraviolet and purple parts of the electromagnetic spectrum. 

\subsubsection{He}

He has a atom number of Z=2 and it's ground state configuration is $1s^2$. The most prominent lines are the 395 nm, 402 nm, 438 nm, 447 nm, 502 nm. These all have in common that they are wavelengths for many different transitions that all have the same relative intensity. Most of our registered trasitions are from the $2p$ shell wwith the one at 502 nm being from the $2s$ to the $3p$ or $3d$ shell. For the other prominent transitions we go to the $9d$, $8s$, $5d$, $4d$, and $4s$. Here we can note than the energy required for a transition to a given shell increases nicely with the primary quantum number. Since we are observing most of our spectral lines in the range 400-500 which is in the purple/blue range of electromagnetic spectrum.  

