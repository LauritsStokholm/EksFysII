\section{Discussion}

\subsection{Element lamps}

We have neglected the hyperfine structure as the database does not take it into acquaintance. 

\subsubsection{Xe}

Xe has Z=54 and its ground state configuration is $1s^2 2s^2 2p^6 3s^2 3p^6 3d^10 4s^2 4p^6 4d^10 5s^2 5p^6$. For simplicity we only noted the two last terms, as the are the electrons with lowest binding energy and are therefore the most likely to be excited. Furthermore, when a sufficient voltage is applied to the lamp, the gas becomes ionized and thus gets a free electron, that it did not have before due to it being a noble gas. Looking at our Xenon spectrum, we see that the most prominent lines with the highest relative intensity are 472 nm, 823 nm, 882 nm, 894 nm, 895 nm. Most of the transitions are from the $6s$ to the $6p$ shell, which is a shift to the subshell one level higher. The 472 nm transitions is however between to shells the $6s$ and the $7p$, which also makes sense since the low wavelengths has more energy, which makes jumps between states with different primary quantum numbers possible, as opposed to jumps between subshells which requires far less energy due to the close splitting between the fine structure energies. It could also be noted that none of our jumps are from the ground state of Xe, but rather from an excited state caused by the voltage over the gas. Furthermore the peaks lies primarily within the 450-480 nm and 800-900 nm range which means that Xe radiates in the blue and infrared part of the spectrum. '

\subsubsection{Hg}

Hg has atomic number Z=80 and has ground state configuration $1s^2 2s^2 2p^6 3s^2 3p^6 3d^10 4s^2 4p^6 4d^10 4f^{14}5s^2 5p^6 5d^10 6s^2 {}^1S_0$,

where ${}^1S_{0}$ is an electronic configuration with total spin $S=0$ and total angular momentum $J=0$.\\

The prominent lines in the spectrum are 318 nm, 405 nm, 435 nm and 545 nm. All of these transitions are from the $6p$ subshell, where the 313 nm and 364 nm 