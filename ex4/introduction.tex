\section{Introduction}

The aim of this excercise is to learn the basics of emission and absorption
spectroscopy. The repport will be split into three sections. First of all,
three light bulbs; a diode-based, a halogen and an energy saving bulb, will be
investigated and compared to the spectra of that of a blackbody. The
temperature of the various light sources will be determined and so will they be
compared to the solar spectrum. 

As the recorded solar spectrum also contains absorption lines (Fraunhofer lines), which are
narrow regions of decreased intensity, that are the result of
photons being absorbed as light passes from the source to the detector. In the
Sun, Fraunhofer lines are a result of gas in the photosphere. By our
recorded solar spectrum, we will determine some of the molecules, which
consequently must be in the outer layer of the sun. 

In addition, theemission spectra from spectral lamps and a laser will be studied.

Look at the line spectra from such lamps and compare with a data base to assign
the initial and final electronic stats for as many lines as possible. 



