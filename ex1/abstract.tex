%Denne rapport er skrevet i foråret d. \today, som den fjerde obligatoriske rapport i en række af fem inden for kurset \emph{Statistik og Databehandling (fysik)}. Der udføres to eksperimenter; et om værdien, variansen og konfidensintervallet af modstanden over en metaltråd af varierende længde. Det andet vedrører en dybere undersøgelse af negative Helium ioners tre tilstande og deres henfald. 
%Resultaterne viste at hypotesen om en ligefrem proportionalitet mellem ledningslængde og modstand ikke kan bortkastes af data, og at temperaturen i ELISA lagerringen har været omkring $\SI{-35}{\degreeCelsius}$ under datasamling af den negative helium ion. Nøgleord i rapporten er \emph{Residualplot}, \emph{Lineær regression} og \emph{Curve--fitting}.  
This paper is written as the first of four mandatory repports during the course \emph{Experimental Physics II}. In this experiment we will be working with the Fresnel relations, and an comparison of theory and experiment is the primorial purpose.


