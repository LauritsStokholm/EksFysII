\documentclass[a4paper, oneside, onecolumn, 11pt]{memoir}
\newenvironment{code}{\captionsetup{type=listing}}{}

%\begin{code}
%    \caption{Den skrevne Pythonkode.
%    \label{kode}}

%    \inputminted[firstline=20, lastline=30,
%        frame=single, framesep=2mm, fontsize=\footnotesize, linenos% Spanning over more than one page!
%    ]{python}{kode.py}
%\end{code}

% Preamble
\usepackage{preamble}

\graphicspath{{./Graphics/}}

\title{Logbook}
\author{Anne Kirstine Knudsen\thanks{email} \and Laurits N. Stokholm\thanks{laurits.stokholm@post.au.dk}}
\date{\today}

\begin{document}
\maketitle
%\tableofcontents

\section{Problem and Aim}
% What is the problem
\begin{itemize}
    \item Set up a Michelson Interferometer
    \item Determination of the expansion coefficient of a piezoelectric element
    \item Study the effect of intensity differences of the interferometer arms
\end{itemize}


\subsection{Research method}
\begin{figure}[h!]
    \centering
    \includegraphics[width=\columnwidth]{michelsonsetup}
    %    \caption{Sketch of a Michelson interferometer. The beam from the laser (1) is aimed at the beam splitter (2) which divides the beam into two partial beams. The two beams are reflected by the mirrors (3). An interference pattern can be observed on the screen/detector (4)}
    \label{fig:michelsonsetup}
\end{figure}

\subsubsection{Planning}
% Start each entry by stating what you are planning to do



\subsubsection{Experimental Equipment Available}
% What is the experimental 
\begin{itemize}
    \item Ruler
    \item Red diode Laser $650 \si{\nano\meter}$
    \item Collimating slits with 5 slits
    \item Polarizer with rotational mount
    \item Polarizer
    \item Collimating lens
    \item Rotational mount
    \item High sensitivity light sensor
    \item PicoScope
\end{itemize}

\subsubsection{Critical issues}
% What are the critical issues?
Intensity of light is half s- and p-polarized.
Allignment of detector and laserbeam.

\subsubsection{Strategy}
% How to carry out the tasks
% Strategy for carry out the experiments
% - which parameter have to be meassured and how to best do it
% - taking reflections into account

\subsection{Setup}
% Make a sketch of the experiment, explain why this setup.

\subsection{Laboratory setyp}
% Does it work as expected
% Did you chose to go differently than originally sketched?
% Add pictures

\subsection{Raw data}
% Screen shots and link to data files

% 1st Meassurement
%slit width 2, 0.5 mm

\subsection{Fast analysis}
% Does the data seem to be valid?
% Do you see any sign of systematic errors?

\subsection{Conclusion}
% What have you learned?
% What has or could be done differently for improving the results?
Her og der og alle vegne, som du kan se på \cref{SourceCode:1}

\begin{code}
	\caption{Caption
    \label{SourceCode:1}}
    \inputminted[firstline=1, lastline=5, frame=single, framesep=2mm, fontsize=\footnotesize, linenos% Spanning over more than one page!
]{python}{kode.py}
\end{code}

\end{document}

