\section{Introduction}
Interferometry is a family of techniques, in which waves are superimposed causing the phenomenon of interference. It lies in the very heart of modern physics and is a method frequently used in the fields of the very large (Most recently in detection of gravitational waves at LIGO), to the domain of quantum interference and particle physics (notably the detection and meassurement of hyperfine structure in line spectra). In addition, the method is of historical importance as it was used to disprove the luminiferous aether (Michelson-Morley experiment 1887) leading to the theory of special relativity. 

\noindent
In this repport we will be looking at two different setups of interferometry: the simple Michelson interferometer where a single beamsplitter is used, followed by the Mach-Zehnder interferometer which uses two beamsplitters.
